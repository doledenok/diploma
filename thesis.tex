\documentclass[oneside,senior,etd]{BYUPhys}

\usepackage{cmap} % Для корректной кодировки в pdf
\usepackage[utf8]{inputenc}
\usepackage{rotating}

\usepackage[russian]{babel}
\usepackage{amsfonts} % Пакеты для математических символов и теорем
\usepackage{amstext}
\usepackage{amssymb}
\usepackage{amsthm}
\usepackage{graphicx} % Пакеты для вставки графики
\usepackage{subfig}
\usepackage{color}
\usepackage[unicode]{hyperref}
\usepackage[nottoc]{tocbibind} % Для того, чтобы список литературы отображался в оглавлении
\usepackage{verbatim} % Для вставок заранее подготовленного текста в режиме as-is
\usepackage{listings}

\newcommand{\sectionbreak}{\clearpage} % Раздел с новой станицы

\usepackage{tikz}
\usepackage{pgfplots}
\usetikzlibrary{arrows,positioning}
\usepackage{adjustbox}

\usepackage{makecell}
\usepackage{booktabs}
\usepackage{boldline}

\usepackage{xcolor}
\usepackage{soul}
\usepackage{url}
\usepackage{multirow}
\usepackage{amsmath}

\usepackage{pifont}
\usepackage{indentfirst} % Делать отступ в начале первого параграфа

% Общие параметры листингов
\lstset{
  %frame=TB,
  showstringspaces=false,
  tabsize=4,
  basicstyle=\linespread{1.0}\tt\small, % делаем листинги компактнее
  breaklines=true,
  texcl=true, % русские буквы в комментах
  captionpos=b,
  aboveskip=\baselineskip,
  commentstyle=\tt
}

\newcommand{\todo}[1]{\textcolor{red}{#1}}

% DEBUG
% \usepackage{showframe}

\Faculty{Факультет вычислительной математики и кибернетики}
\Chair{Кафедра системного программирования}
  % Лаборатория
  % \Lab{~}
\Year{2023}
  \Month{Май}
  \City{Москва}
  \AuthorText{Автор:}
  \Author{Доледенок Максим Вадимович}
  \AuthorEng{Max Doledenok}
  \AcadGroup{428}

  \TitleTop{Разработка средств активного наблюдения за состоянием}
  % Раскомментируйте, если нужны еще строчки названия
  \TitleMiddle{бортовой операционной системы реального времени}
  % \TitleBottom{Третья строка названия}
  % Uncomment if you need English title
  \TitleTopEng{Tools for active monitoring of airborne real-time operating system state}
  %\TitleMiddleEng{Thesis theme, second line}
  % \TitleBottomEng{Thesis theme, third line}

\docname{Выпускная квалификационная работа}
  \Advisor{Камкин Александр Сергеевич}
  \AdvisorDegree{к.ф.-м.н.\\}
  % Раскомментируйте, чтобы добавить научного консультанта
  \Consultant{Чепцов Виталий Юрьевич}
  %\ConsultantDegree{степень, звание\\}

% Закомментируйте, если аннотация не нужна
\Abstract{
  При эксплуатации бортовой операционной системы, предназначенной для применения на космических аппаратах,
  в программно-аппаратном комплексе должно возникать как можно меньше непредвиденных ситуаций. Одна из
  возможностей минимизировать количество таких ситуаций - это предварительная отработка полетного задания
  на программно-аппаратном комплексе в более контролируемых условиях на земле. Для большего контроля над
  системой во время отработки полётного задания требуется регулярно получать телеметрическую информацию
  изнутри системы. То есть получать данные из памяти системы в реальном времени. Также для моделирования
  исключительных ситуаций необходима возможность в реальном времени изменять состояние программного
  обеспечения.

  В данной работе рассматривается метод разработки инструмента для доступа к памяти системы
  на примере операционной системы реального времени JetOS, соответствующей стандарту ARINC-653.
}
% Раскомментируйте, если нужна английская аннотация
\AbstractEng{
  When operating an on-board operating system intended for use on spacecraft, as few unforeseen
  situations as possible should occur in the hardware and software complex. One of the ways to
  minimize the number of such situations is to pre-test a flight task on a hardware and software
  complex in more controlled conditions on the ground. For greater control over the system during
  the flight task, it is required to regularly receive telemetry information from inside the system.
  That is, to receive data from the system memory in real time. Also, to simulate exceptional situations,
  it is necessary to be able to change the state of the software in real time.

This paper discusses the method of developing a tool for accessing system memory using the example
of the JetOS real-time operating system conforming to the ARINC-653 standard.
}

% Раскомментируйте, чтобы написать благодарности
% \Acknowledgments{Благодарности.}

%%%% DON'T change this. It is here because .sty does not support cyrillic cp properly %%%%
\TitlePageText{Титульная страница}
\University{Московский государственный университет имени М.В.Ломоносова}
\GrText{гр.}
\AdvisorText{Научный руководитель}
\ConsultantText{Научный консультант}
\AbstractText{Аннотация}
\AcknowledgmentsText{Благодарности}
\ListingText{Листинг}
\AlgorithmText{Алгоритм}

% Set PDF title and author
\hypersetup{
  pdftitle={Курсовая},
  pdfauthor={Доледенок Максим}
}

\begin{document}
\fixmargins
 \makepreliminarypages

\oneandhalfspace

\newenvironment{compactlist}{
  \begin{list}{{$\bullet$}}{
  \setlength\partopsep{0pt}
  \setlength\parskip{0pt}
  \setlength\parsep{10pt}
  \setlength\topsep{8pt}
  \setlength\itemsep{0pt}
  }
  }{
  \end{list}
}

\pdfbookmark[section]{\contentsname}{toc}
\tableofcontents

\section{Введение}

\subsection{ОСРВ}

Операционные системы реального времени (ОСРВ) -- это операционные системы, одним из важнейших
требований к которым является выполнение поставленных задач за заранее определенное время.
Это отличает их от известных пользовательских операционных систем, где фактор времени не так
важен. ОСРВ используются там, где небольшая временная задержка может привести к существенным
проблемам. Например, в авиакосмической отрасли, на некоторых производствах, в системах
аварийной защиты.

\subsection{ОСРВ JetOS}

В данной работе рассматривается операционная система реального времени(ОСРВ) JetOS, разрабатываемая
в Институте Системного Программирования Российской Академии Наук. ОСРВ JetOS предназначена для
использования в авионике, в частности, в гражданских самолетах и спутниках. JetOS разрабатывается
в соответствии со стандартом ARINC 653, который регламентирует временное и пространственное
разделение ресурсов авиационной ЭВМ и определяет программный интерфейс, которым должно
пользоваться прикладное ПО для доступа к ресурсам ЭВМ. Единицей планирования ресурсов является
раздел, аналог пользовательской программы. Каждый раздел получает как процессорное время, 
так и некоторую часть оперативной памяти.

\subsection{Конфигурация памяти в JetOS}

В JetOS конфигурация памяти является статической, то есть раскладка памяти происходит
на этапе загрузки операционной системы, исходя из заранее описанной пользователем
конфигурации. При динамической конфигурации памяти пользовательские приложения могут
заимствовать области оперативной памяти для временного использования. Но для ОСРВ
это не всегда подходит, потому что сложно обеспечить детерминированность при работе
с памятью. А для ОСРВ JetOS детерминированность является очень важным фактором,
повышающим безопасность и отказоустойчивость операционной системы. Поэтому в JetOS
применяется строго статическая конфигурация памяти. Причем, согласно стандарту ARINC 653,
каждый раздел имеет своё адресное пространство, такое, что никакой другой раздел
не имеет доступа к памяти этого адресного пространства. А ядро ОСРВ JetOS
имеет доступ к памяти всех разделов с такими же правами, что и сами разделы.

\subsection{Инструмент для активного наблюдения за состоянием ОСРВ}

<Краткое описание инструмента, что он может, ещё раз зачем он(как в аннотации), к какой памяти имеет доступ,
доступ к значениям переменных для удобства пользователя>


\section{Цель работы, постановка задач}

Целью данной работы является разработка и интеграция в ОСРВ JetOS
инструмента для активного наблюдения за состоянием ОСРВ. Необходимо отслеживать значения заданных
переменных. 

На основе поставленной цели можно выделить набор задач, который необходимо выполнить в ходе работы:
\begin{itemize}
  \item Необходимо изучить принцип доступа к памяти в ОСРВ JetOS.
  \item Придумать схему работы инструмента.
  \item Реализовать доступ на чтение памяти разделов и ядра ОС. Причем этот доступ должен
осуществляться как по адресу, так и по имени переменной в разделе.
  \item Реализовать возможность изменения памяти разделов и ядра ОС. Причем этот доступ должен
осуществляться как по адресу, так и по имени переменной.
  \item Реализовать возможность читать и изменять переменные сложных типов данных, таких как структуры,
  \item Реализовать интеграционный проект для тестирования .
\end{itemize}

\section{Обзор существующих решений}

\subsection{JTAG}

\subsection{VxWorks}
<Вставить скрин из документации.>

\subsection{Иные}
<Проприетарный инструмент РКК «Энергия» для ОСРВ RTEMS.>


\section{Исследование и простроение решения задачи}

\subsection{Устройство памяти в ОСРВ JetOS}

Основные принципы устройства памяти в ОСРВ JetOS:
\begin{itemize}
  \item платформа имеет одно или несколько ядер процессора;
  \item набор регионов физической памяти общий для всех ядер;
  \item на одной платформе может быть запущено несколько модулей,
  здесь модуль -- сконфигурированная операционная система, предназначенная для выполнения
  на одном или нескольких ядрах процессора на платформе;
  \item память разделяется на виртуальную и физическую;
  \item виртуальная память своя для каждого модуля, физическая -- общая для всех модулей;
  \item виртуальная память разделяется на привилегированную (доступную только ядру)
  и непривилегированную (доступную и ядру, и разделам);
  \item физическая память разделяется на оперативную память и память физических устройств;
  \item и виртуальная, и физическая память делятся на страницы;
  \item размеры страниц одинаковы для виртуальной и физической памяти;
  \item адрес страницы пропорционален ее размеру.
\end{itemize}

\subsection{Схема работы инструмента}
<Картинка из презентации, и её подробное описание>

\subsection{Задание отслеживания переменных}

\subsubsection{Задание статически в файле YAML}
<Скрин примера с комментариями>

\subsubsection{Задание динамически через консоль}
<Формат, пример>

\subsection{Протокол взаимодействия целевой системы и бортового вычислителя}

\subsubsection{Протокол посылки команд бортовому вычислителю}
<Взять из документации, расписать поподробнее>

\subsubsection{Протокол посылки данных целевой системе}
<Взять из документации, расписать поподробнее>

\subsection{Получение адреса переменной по её названию}

\subsubsection{Переменные простых типов}

<Краткое описание ELF>

\subsubsection{Переменные сложных типов}

<Краткое описание DWARF>

\section{Результаты работы}

<Описание сделанного как в задачах>
<Скрины успешного вывода программы>

\section{Заключение}

<Формальные слова>

\nocite{arinc, sanitizer, scons, jetos, allocator}

%\begingroup
\bibliographystyle{gost780u.bst}
\raggedright

\bibliography{references}
%\cite{arinc, scons, jetos, allocator}
%\endgroup


%\bibliographystyle{gost780u.bst} % Для соответствия требованиям об оформлении списка литературы
%\raggedright
%\bibliography{references}

% Раскомментируйте, если нужно приложение
% \appendix

% \cleardoublepage \phantomsection
% \section*{Приложение}
% \addcontentsline{toc}{section}{Приложение}

\end{document}
